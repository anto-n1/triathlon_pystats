% --------------------------------------------------------------------
% RAPPORT MENSUEL ACTIVITE DE TRIATHLON
% RDATE
% --------------------------------------------------------------------

\documentclass[a4paper,french,11pt]{report}

\usepackage[french]{babel}
\usepackage[utf8]{inputenc}
\usepackage{lmodern}
\usepackage{titling}

\usepackage{csquotes} % Pour les double quotes ""
\usepackage{microtype}

% Permet de faire des listes
\usepackage{blindtext}
\usepackage{enumitem}

% Permet d'utiliser les légendes sur les photos et tableaux 
\usepackage{caption}
\usepackage{subcaption}
\usepackage{wrapfig}

\usepackage{xcolor}
\usepackage[hmargin=1cm,vmargin=1.5cm]{geometry}

\usepackage{graphicx}
\graphicspath{{images/}}

\usepackage{titlesec}
\titleformat{\chapter}[display]{\Huge\bfseries}{}{0pt}{\thechapter.\ }
\titleformat{name=\chapter,numberless}[display]{\Huge\bfseries}{}{0pt}{}

% Footer
%\usepackage{fancyhdr}

%\pagestyle{fancy}
%\fancyhf{}

%\fancyfoot[LE,LO]{Rapport triathlon}
%\fancyfoot[CE,CO]{\thepage}
%\fancyfoot[RE,RO]{RMONTH}

% Une ligne en bas mais pas en haut
%\renewcommand{\headrulewidth}{0.5pt}
%\renewcommand{\footrulewidth}{0.5pt}

\usepackage{hyperref}
\hypersetup{
    hyperfigures = true,
    colorlinks = true,
    linkcolor=black,
    urlcolor=black,
    citecolor=black
}

% Réduire espace avant chapitre
\titlespacing*{\chapter}{0pt}{-20pt}{40pt}


% --------------------------------------------------------------------
% Definitions de commandes
% --------------------------------------------------------------------
\newcommand{\HRule}[1]{\rule{\linewidth}{#1}} 	% Ligne horizontale

\makeatletter							% Titre
\def\printtitle{%						
    {\centering \@title\par}}
\makeatother									

\makeatletter							% Auteur
\def\printauthor{%					
    {\centering \large \@author}}				
\makeatother							

% --------------------------------------------------------------------
% Métadonnées
% --------------------------------------------------------------------
\title{	\normalsize \textsc{\uppercase{ironman project}}
		 	\\[1.2cm]
			\HRule{0.7pt} \\
			\vspace{0.5cm}
			\LARGE \textbf{\uppercase{statistiques triathlon}}	% Titre
			\HRule{1pt} \\
			\vspace{1.7cm}
		}

\author{
		\textbf{RNAME}\\
}

% ------------------------------------------------------------------------------
% Debut du document
% ------------------------------------------------------------------------------
\begin{document}

% ------------------------------------------------------------------------------
% Page de titre
% ------------------------------------------------------------------------------

\thispagestyle{empty}		% Supprimer numéro de page sur cette page

\printtitle					% Afficher le titre défini ci-dessus
\printauthor				% Afficher l'auteur défini ce-dessus
\vspace{1cm}

\begin{center}

\LARGE{\textbf{RDATE}}\\
\vspace{4cm}
\includegraphics[height=6cm]{RIMAGE}\\

\end{center}

\newpage

% ------------------------------------------------------------------------------
% Sommaire
% ------------------------------------------------------------------------------
%\tableofcontents

% ------------------------------------------------------------------------------
% Statistiques globales tous sports confondus
% ------------------------------------------------------------------------------

\section*{Statistiques tous sports confondus}
\thispagestyle{empty}


\noindent{\textbullet Nombre d'entraînements : \textbf{RACTIVITIES-NUMBER} - ratio de \textbf{RRATIO entraînement/jour}}\\
\textbullet Distance totale parcourue : \textbf{RTOTAL-DISTANCE km} - avec \textbf{RELEVATION m} de dénivelé positif\\
\textbullet Temps total d'activité : \textbf{RDURATION} \\
\textbullet Vitesse moyenne : \textbf{RAVERAGE-SPEED km/h} \\
\textbullet Fréquence cardiaque moyenne : \textbf{RAVERAGE-HR bpm} - maximale atteinte \textbf{RMAX-HR bpm}\\
\textbullet VO2max moyenne approximative : \textbf{RAVERAGE-VO2 ml/kg/min} - maximale atteinte \textbf{RMAX-VO2 ml/kg/min}\\

\begin{figure}[!ht]
\begin{subfigure}{.49\linewidth}

\centering

%\includegraphics[width=0.5\textwidth]{graphs/repartition_temps.png}
\includegraphics[width=\linewidth]{graphs/repartition_temps.png}
\caption*{Répartition du temps passé}

\end{subfigure}\hfill
\begin{subfigure}{.49\linewidth}

\centering

\includegraphics[width=\linewidth]{graphs/repartition_distance.png}
\caption*{Répartition des distances parcourues}

\end{subfigure}

\end{figure}


\begin{figure}[!ht]
\begin{subfigure}{.49\linewidth}

\centering

\includegraphics[width=\linewidth]{graphs/repartition_nombre_activites.png}
\caption*{Répartition du nombre d'activités}

\end{subfigure}\hfill
\begin{subfigure}{.49\linewidth}

\centering

\includegraphics[width=\linewidth]{graphs/repartition_place.png}
\caption*{Répartition des lieux d'activité (running et cyclisme)}

\end{subfigure}

\end{figure}

\newpage

% ------------------------------------------------------------------------------
% Statistiques pour le cyclisme
% ------------------------------------------------------------------------------

\section*{Statistiques cyclisme}
\thispagestyle{empty}


\noindent{\textbullet Nombre d'entraînements : \textbf{RCACTIVITIES-NUMBER} - ratio de \textbf{RCRATIO entraînement/jour}}\\
\textbullet Distance totale parcourue : \textbf{RCTOTAL-DISTANCE km} - avec \textbf{RCELEVATION m} de dénivelé positif\\
\textbullet Temps total d'activité : \textbf{RCDURATION} \\
\textbullet Vitesse moyenne : \textbf{RCAVERAGE-SPEED km/h} \\
\textbullet Fréquence cardiaque moyenne : \textbf{RCAVERAGE-HR bpm} - maximale atteinte \textbf{RCMAX-HR bpm}\\
\textbullet VO2max moyenne approximative : \textbf{RCAVERAGE-VO2 ml/kg/min} - maximale atteinte \textbf{RCMAX-VO2 ml/kg/min}\\

% ------------------------------------------------------------------------------
% Statistiques pour le running
% ------------------------------------------------------------------------------

\section*{Statistiques running}
\thispagestyle{empty}


\noindent{\textbullet Nombre d'entraînements : \textbf{RPACTIVITIES-NUMBER} - ratio de \textbf{RPRATIO entraînement/jour}}\\
\textbullet Distance totale parcourue : \textbf{RPTOTAL-DISTANCE km} - avec \textbf{RPELEVATION m} de dénivelé positif\\
\textbullet Temps total d'activité : \textbf{RPDURATION} \\
\textbullet Vitesse moyenne : \textbf{RPAVERAGE-SPEED km/h}\\
\textbullet Fréquence cardiaque moyenne : \textbf{RPAVERAGE-HR bpm} - maximale atteinte \textbf{RPMAX-HR bpm}\\
\textbullet VO2max moyenne approximative : \textbf{RPAVERAGE-VO2 ml/kg/min} - maximale atteinte \textbf{RPMAX-VO2 ml/kg/min}\\

% ------------------------------------------------------------------------------
% Statistiques pour la natation
% ------------------------------------------------------------------------------

\section*{Statistiques natation}
\thispagestyle{empty}


\noindent{\textbullet Nombre d'entraînements : \textbf{RNACTIVITIES-NUMBER} - ratio de \textbf{RNRATIO entraînement/jour}}\\
\textbullet Distance totale parcourue : \textbf{RNTOTAL-DISTANCE km}\\
\textbullet Temps total d'activité : \textbf{RNDURATION} \\
\textbullet Vitesse moyenne : \textbf{RNAVERAGE-SPEED km/h}

% ------------------------------------------------------------------------------
% Statistiques pour le renforcement musculaire
% ------------------------------------------------------------------------------

\section*{Statistiques renforcement musculaire}
\thispagestyle{empty}


\noindent{\textbullet Nombre d'entraînements : \textbf{RMACTIVITIES-NUMBER} - ratio de \textbf{RMRATIO entraînement/jour}}\\
\textbullet Temps total d'activité : \textbf{RMDURATION} \\
\textbullet Fréquence cardiaque moyenne : \textbf{RMAVERAGE-HR bpm} - maximale atteinte \textbf{RMMAX-HR bpm}\\

% ------------------------------------------------------------------------------
% Fin du document
% ------------------------------------------------------------------------------
\end{document}
